%!TEX program = xelatex
\documentclass[a4paper, oneside, 12pt]{book}


% =========================================================
% 1. НАСТРОЙКИ СТРАНИЦЫ И ШРИФТОВ
% =========================================================
\usepackage{geometry}
\geometry{top=1.5cm, bottom=1.5cm, left=1.5cm, right=1.5cm}

\usepackage{fontspec}
\usepackage{polyglossia}

\setdefaultlanguage{russian}
\setotherlanguage{english}

% --- ШРИФТЫ ---
% Используем Roboto и JetBrains Mono. Если их нет, TeX выдаст ошибку.
% В таком случае замените на Arial / Courier New или установите шрифты.
\setmainfont{Roboto}[
    UprightFont = *-Regular,
    BoldFont = *-Bold,
    ItalicFont = *-Italic
]
\setsansfont{Roboto}[
    UprightFont = *-Regular,
    BoldFont = *-Bold,
    ItalicFont = *-Italic
]
\setmonofont{JetBrains Mono}[
    Scale=0.9,
    Contextuals=Alternate
]

% Фикс для кириллицы в Polyglossia (обязательно!)
\newfontfamily\cyrillicfont{Roboto}[UprightFont=*-Regular, BoldFont=*-Bold, ItalicFont=*-Italic]
\newfontfamily\cyrillicfontsf{Roboto}[UprightFont=*-Regular, BoldFont=*-Bold, ItalicFont=*-Italic]
\newfontfamily\cyrillicfonttt{JetBrains Mono}[Scale=0.9]

% =========================================================
% 2. ЦВЕТОВАЯ ПАЛИТРА (Nord Theme)
% =========================================================
\usepackage{xcolor}

% Основные цвета
\definecolor{nordDark}{HTML}{2E3440}    % Текст
\definecolor{nordLight}{HTML}{ECEFF4}   % Фон страницы (если нужно) или баджей
\definecolor{nordCodeBg}{HTML}{F7F9FB}  % Фон кода (очень светлый)
\definecolor{nordBorder}{HTML}{E5E9F0}  % Рамки

% Акценты
\definecolor{nordBlue}{HTML}{5E81AC}    % Определения, Ссылки
\definecolor{nordOrange}{HTML}{D08770}  % Заметки
\definecolor{nordRed}{HTML}{BF616A}     % Важно / Ошибки
\definecolor{nordPurple}{HTML}{B48EAD}  % Резюме / Итоги
\definecolor{nordGreen}{HTML}{A3BE8C}   % Советы (опционально)

\color{nordDark} % Установка основного цвета текста

% =========================================================
% 3. ССЫЛКИ (HYPERREF)
% =========================================================
\usepackage[hidelinks]{hyperref}
\hypersetup{
    colorlinks=true,        % Красить текст, а не делать рамку
    linkcolor=nordBlue,     % Цвет внутренних ссылок (toc, ref)
    urlcolor=nordBlue,      % Цвет внешних ссылок
    citecolor=nordBlue,
    pdftitle={C++ Notes},
    pdfauthor={Author}
}

% =========================================================
% 4. БЛОКИ И КОД (TCOLORBOX + MINTED)
% =========================================================
\usepackage[most, minted]{tcolorbox}

% --- Общие стили (вынесены, чтобы избежать ошибок) ---
\tcbset{
    % Стиль для кода C++
    cppstyle/.style={
        listing engine=minted,
        sharp corners,
        minted language=cpp,
        colback=nordCodeBg,
        colframe=nordBorder,
        coltitle=nordDark,
        fonttitle=\bfseries,
        enhanced,   % Обязательно для breakable
        breakable,  % <--- ДОБАВИТЬ ЭТУ СТРОКУ
        boxrule=0.5pt,
        arc=6pt,
        left=8mm, right=4mm, top=3mm, bottom=3mm,
        listing only,
        enhanced,
        minted options={
            fontsize=\small,
            breaklines,
            autogobble,
            linenos,
            numbersep=3mm,
            style=friendly % Варианты: friendly, xcode, pastie
        }
    },
    % Стиль для "кнопки" ссылки
    badgestyle/.style={
        on line,
        arc=3pt,
        colback=nordLight,
        colframe=nordLight,
        boxrule=0pt,
        boxsep=0pt,
        left=3pt, right=3pt, top=2pt, bottom=2pt,
        fontupper=\sffamily\footnotesize\bfseries
    }
}

% 1. Окружение для кода
\newtcblisting{cppcode}[1][]{cppstyle, #1}

% 2. Макрос для красивой ссылки на стандарт
% Использование: \stdref{URL}{Название}
\newtcbox{\stdlinkbox}{badgestyle}
\newcommand{\stdref}[2]{%
    \href{#1}{\stdlinkbox{\textcolor{nordBlue}{#2 $\nearrow$}}}%
}

\newtcblisting{pythoncode}[2][]{
    listing engine=minted,       % Используем движок minted
    minted language=python,      % Язык - Python
    listing only,                % Показываем только код (без результата выполнения)
    title={#2},                  % Заголовок (аргумент 2)
    % --- Оформление ---
    colback=black!5!white,       % Цвет фона (очень светло-серый)
    colframe=blue!50!black,      % Цвет рамки (темно-синий)
    coltitle=white,              % Цвет текста заголовка
    fonttitle=\bfseries,         % Жирный шрифт заголовка
    % --- Настройки подсветки (minted) ---
    minted options={
        linenos,                 % Номера строк
        breaklines,              % Перенос длинных строк
        tabsize=4,               % Размер табуляции
        fontsize=\small,         % Размер шрифта кода
    },
    % --- Эстетика рамки ---
    enhanced,                    % Включаем расширенные настройки рисования
    drop shadow,                 % Тень
    #1                           % Сюда пойдут дополнительные опции при вызове
}
% 3. Определение (Definition) - Синий
\newtcolorbox{definition}[2][]{
    colback=white,
    colframe=nordBlue,
    coltitle=nordBlue,
    title={#2},
    fonttitle=\bfseries\sffamily,
    sharp corners, rounded corners=southeast, rounded corners=northeast,
    arc=4pt,
    boxrule=0pt,
    leftrule=3pt,
    left=4mm, right=4mm, top=2mm, bottom=2mm,
    parbox=false,
    enhanced,
    frame hidden,
    borderline west={3pt}{0pt}{nordBlue},
    #1
}

% 4. Заметка (Note) - Оранжевый
\newtcolorbox{note}[1][]{
    colback=nordOrange!10,
    colframe=nordOrange,
    boxrule=0pt,
    arc=6pt,
    left=4mm, right=4mm, top=3mm, bottom=3mm,
    parbox=false,
    title={\textbf{На заметку}},
    coltitle=nordOrange!40!black,
    detach title,
    before upper={\tcbtitle\par\vspace{3pt}},
    #1
}

% 5. Важно (Important) - Красный
\newtcolorbox{important}[1][]{
    colback=nordRed!8,
    colframe=nordRed,
    boxrule=0pt,
    leftrule=3pt,
    arc=4pt,
    left=4mm, right=4mm, top=3mm, bottom=3mm,
    parbox=false,
    title={\textbf{\textsc{Важно!}}},
    coltitle=nordRed,
    fonttitle=\bfseries\sffamily,
    enhanced,
    frame hidden,
    borderline west={3pt}{0pt}{nordRed},
    #1
}

% 6. Резюме (Summary) - Фиолетовый
\newtcolorbox{summary}[1][]{
    colback=nordPurple!10,
    colframe=nordPurple,
    boxrule=0pt,
    arc=6pt,
    left=4mm, right=4mm, top=4mm, bottom=4mm,
    parbox=false,
    title={\centering\textbf{Резюме раздела}},
    coltitle=nordPurple!40!black,
    detach title,
    before upper={\tcbtitle\par\vspace{5pt}{\color{nordPurple!50}\hrule height 1pt}\vspace{5pt}},
    #1
}

% =========================================================
% 5. ПСЕВДОКОД И ТИПОГРАФИКА
% =========================================================
\usepackage[ruled, vlined, linesnumbered]{algorithm2e}
\SetAlFnt{\small\sffamily}
\SetKwComment{Comment}{// }{}
\newcommand\mycommfont[1]{\footnotesize\ttfamily\textcolor{gray}{#1}}
\SetCommentSty{mycommfont}
\setlength{\algomargin}{1.5em}

\usepackage{parskip} % Отступы между абзацами
\usepackage{titlesec}
\usepackage{enumitem}
\usepackage{import}
\usepackage{tikz}
\usetikzlibrary{shapes.geometric, arrows, positioning}

% Заголовки (Sans Serif)
\titleformat{\section}{\Large\bfseries\sffamily\color{nordDark}}{}{0em}{}
\titleformat{\subsection}{\large\bfseries\sffamily\color{nordDark!80}}{}{0em}{}


% =========================================================
% НАЧАЛО ДОКУМЕНТА
% =========================================================
\begin{document}
% =========================================================
% ОБЛОЖКА (COVER PAGE) - NORD STYLE
% =========================================================
\begin{titlepage}
    % Убираем отступы для фона на всю страницу
    \newgeometry{left=0cm, right=0cm, top=0cm, bottom=0cm}
    
    \begin{tikzpicture}[remember picture, overlay]
        % 1. Левая темная панель (nordDark)
        \fill[nordDark] (current page.north west) rectangle ([xshift=7cm]current page.south west);
        
        % 2. Тонкая разделительная линия (nordBlue)
        \fill[nordBlue] ([xshift=7cm]current page.north west) rectangle ([xshift=7.2cm]current page.south west);
        
        % 3. Фоновый декоративный элемент "C++" (полупрозрачный)
        \node[text=nordLight, opacity=0.05, rotate=0, anchor=south east] 
            at ([xshift=-1cm, yshift=2cm]current page.south east) 
            {\fontsize{200}{200}\bfseries\sffamily ++};

        % --- КОНТЕНТ В ЛЕВОЙ ПАНЕЛИ ---
        \node[anchor=north west, text width=6cm, align=flush left] 
            at ([xshift=0.8cm, yshift=-3cm]current page.north west) {
            
            \vspace{1cm}
            % Логотип или название вуза
            {\color{nordBlue}\sffamily\bfseries\Large HSE \par}
            \vspace{0.2cm}
            {\color{nordLight}\sffamily\small Faculty of Computer Science \par}
            
            \vspace{10cm}
            
            % Автор
            {\color{nordBlue}\sffamily\bfseries\small AUTHOR \par}
            {\color{nordLight}\sffamily\Large Your Name \par}
            
            \vspace{1cm}
            
            % Курс / Семестр
            {\color{nordBlue}\sffamily\bfseries\small COURSE \par}
            {\color{nordLight}\sffamily Advanced C++ \par}
            {\color{nordLight}\sffamily Fall 2023 \par}
        };
    \end{tikzpicture}

    % --- КОНТЕНТ В ПРАВОЙ ЧАСТИ ---
    \vspace{6cm}
    \hspace{7.5cm} % Сдвиг вправо, чтобы не наехать на синюю полосу
    \begin{minipage}{11cm}
        % ЗАГОЛОВОК
        {\fontsize{50}{60}\bfseries\sffamily\color{nordDark} 
        Конспект\\[0.2em] 
        углубленного курса\\[0.2em] 
        по C++ \par}
        
        \vspace{0.5cm}
        % Линия под заголовком
        {\color{nordBlue}\hrule height 2pt width 5cm} 
        \vspace{0.8cm}
        
        
        \vspace{3cm}
        
        
    \end{minipage}
    % Возвращаем нормальные отступы для остального документа
    \restoregeometry
\end{titlepage}

\tableofcontents
\part{Лекция 01 – Введение. Память}
\import{1/}{main_1.tex}
\part{Лекция 02 – Динамическая память. Move семантика}
\import{2/}{main_2.tex}
\part{Лекция 03 – Продолжаем про move семантику}
\import{3/}{main_3.tex}
\part{Лекция 04 – Типы. Шаблоны}
\import{4/}{main_4.tex}
\part{Лекция 05 – Ошибки. Исключения. Noexcept}
\import{5/}{main_5.tex}
\part{Лекция 06 – Паттерны. ODR}
\import{6/}{main_6.tex}
\part{Лекция 07 – Метапрограммирование}
\import{7/}{main_7.tex}
\part{Лекция 08 – Baby thread}
\import{8/}{main_8.tex}
\part{Лекция 09 – Condition variable}
\import{9/}{main_9.tex}
\part{Лекция 10 – Advanced thread}
\import{10/}{main_10.tex}
\part{Лекция 11 – lock free}
\import{11/}{main_11.tex}
\part{Лекция 12 – корутины}
\import{12/}{main_12.tex}
\end{document}
